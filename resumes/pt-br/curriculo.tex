% GitHub Repo and Documentation: https://github.com/celiobjunior/resume-template
% Copyright © 2025 Celio B Junior. All rights reserved.
% 
% Licensed under the Apache License, Version 2.0 (the "License");
% you may not use this file except in compliance with the License.
% You may obtain a copy of the License at
%
%     http://www.apache.org/licenses/LICENSE-2.0
%
% This template follows best practices from README.md

% Início do documento LaTeX: define tipo e formato do documento
% a4paper = tamanho A4, 10pt = tamanho base da fonte
\documentclass[a4paper,10pt]{article}

% --- PACOTES ---
\usepackage[utf8]{inputenc}
\usepackage[T1]{fontenc}
\usepackage[portuguese]{babel}
\usepackage{geometry}
\usepackage{parskip}
\usepackage{hyperref}
\usepackage{titlesec}
\usepackage[sfdefault]{carlito}
\usepackage[utf8]{inputenc}
% Para quebras de linha em URLs longas (não use \href{}, use \url{} para URLs longas)
% \usepackage{xurl}

% --- CONFIGURAÇÃO DO DOCUMENTO ---
% Define margens da página para maximizar espaço de conteúdo
\geometry{top=1.0cm, bottom=1.0cm, left=1.0cm, right=1.0cm}

% Remove números de página e cabeçalhos para visual limpo do currículo
\pagestyle{empty}

% Metadados do PDF - personalize com suas informações
\hypersetup{
    pdftitle={CV Seu Nome},
    pdfauthor={Seu Nome},
    colorlinks=true,
    linkcolor=black,
    urlcolor=black,
    citecolor=black,
    bookmarksdepth=1 
}

% Desabilita numeração das sections
\setcounter{secnumdepth}{0}

% Formata os cabeçalhos de cada section e coloca uma linha em baixo
\titleformat{\section}
{\Large\bfseries}
{}
{0em}
{}
[\titlerule\vspace{0.5ex}]

% --- INÍCIO DO DOCUMENTO ---
\begin{document}

% --- CABEÇALHO ---
% Substitua com suas informações pessoais
\begin{center}
    {\LARGE \textbf{Lucas Gregorio Martins}} 
    \\ [0.1cm]
    +55 11 941935431
    {\textbullet}
    Email: \href{luucasgregorio@gmail.com}{luucasgregorio@gmail.com} 
    {\textbullet}
    \href{https://www.linkedin.com/in/lucas--gregorio/}{linkedin.com/in/lucas--gregorio} 
    {\textbullet}
    \href{https://github.com/llucasgregorio}{github.com/llucasgregorio}
\end{center}

% --- SEÇÕES ---

\section{Resumo} 

    Engenheiro DevOps/SRE especializado em ambientes de alta disponibilidade e modernização de infraestrutura em nuvem. Expertise nos ecossistemas Azure, AWS, Kubernetes, Observabilidade, CI/CD, Terraform, Ansible. Histórico comprovado de refatoração de arquiteturas para ganhos de performance, escalabilidade e otimização de custos.


\section{Experiência}
    % Liste as suas experiência de trabalho ou acadêmicas, mais recente primeiro
    \subsection*{\texorpdfstring{
            \textbf{TOTVS S.A} \hfill Híbrido São Paulo
        }{
            Nome da Empresa -- Localização
        }}
    \textit{Engenheiro DevOps Pleno \hfill 04/2022 - Emprego Atual}
        \begin{itemize} 
            % Foque em conquistas, não apenas na parte técnica - use o método STAR (Situação, Tarefa, Ação, Resultado)
            \item Alcancei redução de custos em nuvem Azure através de otimizações de infraestrutura resultando em mais de 20\% anual de saving.
            
            \item Implementei pipelines de CI/CD com integração de análises de segurança SAST e DAST utilizando a ferramenta Checkmarx reduzindo em mais de 60\% as vulnerabilidades críticas nos produtos.
            
            \item Contribuí e idealizei um projeto para acelerar a implantação do produto em novos clientes por meio da padronização de pipelines e da definição de novos fluxos de entrega, resultando na redução do tempo de go-live.
            
            \item Estruturei pipeline de testes automatizados de front-end com a ferramenta Cypress integrando-o ao processo de CI/CD, aumentando a confiabilidade das entregas e otimizando fluxo de trabalho do time de produto.
            
            \item Otimizei a arquitetura de proxy reverso, com a implementação de WAF e configurações robustas no NGINX, viabilizando a retomada da comercialização de um produto legado do segmento.
    
        \end{itemize}

    \subsection*{\texorpdfstring{
        	\textbf{TQI} \hfill Híbrido São Paulo
        }{
            Nome da Empresa -- Localização
        }}
    \textit{Analista de infraestrutura \hfill 04/2020 - 04/2022}
        \begin{itemize}
            \item Implementei infraestrutura como código com Terraform para provisionamento automatizado em ambientes VMware, promovendo padronização, redução de erros manuais e agilidade nas entregas de servidores.
            
            \item Realizei análise e troubleshooting de incidentes em ambientes Linux, implementando soluções definitivas para melhoria de estabilidade, performance e disponibilidade dos serviços.
            
            \item Administrei ambientes do Microsoft 365, com foco em governança, segurança e suporte ao Exchange Online, garantindo continuidade operacional e conformidade das comunicações corporativas.
            
            \item Estruturei monitoramento proativo da infraestrutura com Zabbix, implementando alertas inteligentes e métricas críticas para redução de downtime e melhoria do MTTR.
            
            \item Implementei políticas e rotinas automatizadas de backup com Iperius Backup, garantindo estratégia de recuperação de desastres (DR) e proteção eficiente dos dados críticos.
            
        \end{itemize}

    \subsection*{\texorpdfstring{
        	\textbf{LINX} \hfill Presencial São Paulo
        }{
            Nome da Empresa -- Localização
        }}
    \textit{Analista de suporte \hfill 02/2020 - 04/2020}
        \begin{itemize}
            \item Executei queries em banco de dados utilizando PL/SQL para análise de causa raiz e suporte à resolução de incidentes.
            
            \item Prestei suporte e sustentação a sistemas de ponto de venda (POS), assegurando alta disponibilidade das operações comerciais e rápida resolução de incidentes em ambientes de varejo.
            
            
        \end{itemize}


\section{Habilidades}
    % Mantenha esta seção concisa e use o máximo de palavras chaves
    % que fazem sentido para a vaga que você deseja.
    \begin{itemize}
        \item \textbf{Plataformas Cloud:} Azure, AWS, Google Cloud Platform (GCP).
        
        \item \textbf{DevOps \& Infraestrutura:} Kubernetes, Helm, Docker, Terraform, Ansible, Bash, Powershell, Python, Jenkins, Azure DevOps, SQL Server, Oracle
        
        \item \textbf{Monitoramento \& Sistemas Operacionais:} Grafana, Prometheus, Zabbix, Linux, Windows Server, VMware.
        
        \item \textbf{Soft Skills:} Mentoria, resolução de problemas técnicos, comunicação clara com clientes e gerencia .
        \item \textbf{Idiomas:} Português (Nativo), Inglês (Intermediário Superior)
    \end{itemize}

% ------

% !!!!!!!!!!!!!!!!!!!!!!
% Mude esta section para o começo, depois do header e antes
% das suas experiências se você está procurando sua primeira
% vaga ou algum estágio. Do contrário, deixe como está.
% !!!!!!!!!!!!!!!!!!!!!!

\section{Educação}
    % Formação mais recente primeiro
    \subsection*{\texorpdfstring{
            \textbf Centro Universitário FAM \hfill Graduação EAD
        }{
            Nome da Sua Universidade (Educação) -- Localização
        }}
    \textit{Análise e Desenvolvimento de sistemas \hfill Data de Término 12/2026}
    
    \subsection*{\texorpdfstring{
            \textbf Senai de informática \hfill Presencial
        }{
            Nome da Sua Universidade (Educação) -- Localização
        }}
    \textit{Técnico em Desenvolvimento Fullstack \hfill 06/2019 - 12/2019}

\end{document}
